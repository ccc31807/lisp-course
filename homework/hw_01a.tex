\documentclass{article}

\usepackage[top=1in]{geometry}
\usepackage{listings}
\usepackage{multicol}

\title{Homework, Week 1, part A}
\author{lightprogramming.org}
\date{\today}

\begin{document}
\maketitle{}

The software development process has four steps. This week's homework will walk you through all four steps. The steps are: (1) analysis, (2) design, (3) implementation, and (4) testing. The process repeats all four steps, one after another, many times before the software is complete.

The purpose of analysis is to develop the \textbf{functional requirements} of the software -- what the software is supposed to do. Suppose you were to develop an addition test. Write down everything you can think of that the software should do. This might include things like (a) make up problems, (b) check answers, and (c) count the correct answers.

One common way of determining requirements is to complete the process by hand, making careful notes about what you did for each step. Think about a math test you took recently. What did you do while taking the test? Now, pretend you are the test-giver, and think of the steps you had to take to give and grade the test. Make a list of these steps, and then turn these steps into a list of requirements. 

The purpose of design is to think about how the software will actually be written. Taking your functional requirements from the previous answer, develop a design for your addition test. This might include such things as (a) start the test, (b) stop the test, (c) tell when an answer is correct, and (d) correct wrong answers.

One way to think about design is in terms of \textit{functions}. There are other ways -- objected-oriented development is extremely popular and widespread. For many years, procedural development was universally taught in software development courses. Howeve, we will think about the design in terms of \textit{functions}. A ``function'' is a piece of code that accepts one or more inputs and returns one (or perhaps more than one) output. An example would be an addition problem -- it takes two inputs and returns one output, which is the sum of the inputs. Try to think of a design composed of functions.

Make sure that you think hard about your answers and that you write them down. You will use these answers for the rest of the week when we implement the addition test.

%\lstset{language=Lisp}
%\begin{multicols}{2}
%\begin{lstlisting}
%(+)
%(+ 1)
%(+ 1 2)
%(+ 1 2 3)
%(+ 1 2 3 4)
%
%(-) ;this is an error
%(- 4)
%(- 4 3)
%(- 4 3 2)
%(- 4 3 2 1)
%
%(*)
%(* 1)
%(* 1 2)
%(* 1 2 3)
%(* 1 2 3 4)
%
%(/)
%(/ 4)
%(/ 4 3)
%(/ 4 3 2)
%(/ 4 3 2 1)
%
%;;do these give the same answer?
%(/ 7 21)
%(/ 7.0 21)
%(/ 7 21.0)
%
%(expt 2 0)
%(expt 2 1)
%(expt 2 2)
%(expt 2 3)
%(expt 2 4)
%(expt 2 5)
%(expt 100 2)
%(expt 2 100)
%;;this will impress you!
%(expt 99 99)
%\end{lstlisting}
%\end{multicols}

\end{document}
