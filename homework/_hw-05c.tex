\documentclass{article}

\usepackage{listings}
\usepackage{multicol}
\usepackage{url}

\title{Homework, Week 7}
\date{December 13, 2013}

\begin{document}
\maketitle{}

\paragraph{}This homework requires you to write a program that encrypts and decrypts text. The encryption method is a substitution cipher. It works by substituting the plain text character for another character. There are three versions. The first version is a simple substitution. The code is a one letter character. For example, if the key is 'e', each letter is shifted five places. For example, 'a' becomes 'f', 'b' becomes 'g', and 'z' becomes 'e'.

\paragraph{}The second version uses a multi character key. For example, if the key were 'maze', the first character would be shifted 13 places, the second character would be shifted a place, the third character would be shifted 26 places, and the fourth character would be shifted five places, and the pattern would repeat.

\paragraph{}The third version uses a multi character continuous key. For example, if the key were 'maze', the first four characters of the plain text would be shifted as in the second version. The fifth character would be shifted the value of the FIRST character of the plain text, the character after that would be shifted the value of the SECOND character of the plain text, and so on.

Here is an example.

\begin{verbatim}
;; Dribble of #<IO TERMINAL-STREAM> started on 2013-12-13 07:32:13.
#<OUTPUT BUFFERED FILE-STREAM CHARACTER #P"cipher-4.txt">
[8]> (encrypt-1)
Enter a line of text to be encrypted: "We attack at dawn."
Enter text to be your key: e
Continuous key? (Y|N): n
Text: [(87 69 65 84 84 65 67 75 65 84 68 65 87 78)], Full-Key: [(5 5 5 5 5 5 5 5 5 5 5 5 5 5)],
   Continuous: NIL, Cont-Key: (5 23 5 1 20 20 1 3 11 1 20 4 1 23 14)
"BJFYYFHPFYIFBS"
[9]> (decrypt-1)
Enter a line of text to be decrypted: "BJFYYFHPFYIFBS"
Enter text that is your key: e
Continuous key? (Y|N): n
Text: [(66 74 70 89 89 70 72 80 70 89 73 70 66 83)], Full-Key: [(5 5 5 5 5 5 5 5 5 5 5 5 5 5)],
   Continuous: NIL, Cont-Key: (5 2 10 6 25 25 6 8 16 6 25 9 6 2 19)
"WEATTACKATDAWN"
[10]> (encrypt-1)
Enter a line of text to be encrypted: "We attack at dawn."
Enter text to be your key: maze
Continuous key? (Y|N): n
Text: [(87 69 65 84 84 65 67 75 65 84 68 65 87 78)], Full-Key: [(13 1 26 5 13 1 26 5 13 1 26 5 13 1 26 5)], 
   Continuous: NIL, Cont-Key: (13 1 26 5 23 5 1 20 20 1 3 11 1 20 4 1 23 14)
"JFAYGBCPNUDFJO"
[11]> (decrypt-1)
Enter a line of text to be decrypted: "JFAYGBCPNUDFJO"
Enter text that is your key: maze
Continuous key? (Y|N): n
Text: [(74 70 65 89 71 66 67 80 78 85 68 70 74 79)], Full-Key: [(13 1 26 5 13 1 26 5 13 1 26 5 13 1 26 5)], 
   Continuous: NIL, Cont-Key: (13 1 26 5 10 6 1 25 7 2 3 16 14 21 4 6 10 15)
"WEATTACKATDAWN"
[12]> (encrypt-1)
Enter a line of text to be encrypted: We attack at dawn.
Enter text to be your key: maze
Continuous key? (Y|N): y
Text: [(87 69 65 84 84 65 67 75 65 84 68 65 87 78)], Full-Key: [(13 1 26 5 13 1 26 5 13 1 26 5 13 1 26 5)], 
   Continuous: T, Cont-Key: (13 1 26 5 23 5 1 20 20 1 3 11 1 20 4 1 23 14)
"JFAYQFDEUUGLXH"
[13]> (decrypt-1)
Enter a line of text to be decrypted: JFAYQFDEUUGLXH
Enter text that is your key: maze
Continuous key? (Y|N): y
Text: [(74 70 65 89 81 70 68 69 85 85 71 76 88 72)], Full-Key: [(13 1 26 5 13 1 26 5 13 1 26 5 13 1 26 5)], 
   Continuous: T, Cont-Key: (13 1 26 5 10 6 1 25 17 6 4 5 21 21 7 12 24 8)
"WEATGZCFDOCGCM"
[14]> (dribble)
;; Dribble of #<IO TERMINAL-STREAM> finished on 2013-12-13 07:35:55.
\end{verbatim}

\paragraph{}In my code, the third version of \texttt{(decrypt-1)} isn't working correctly. I will fix this soon and post the code.

\paragraph{}This homework is a multi-week homework assignment. Work on this a piece at a time, and apply everything you have learned so far. I have posted by code, but looking at my solution before you have sweated blood is cheating.
\end{document}
