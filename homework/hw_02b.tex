\documentclass{article}

\usepackage{listings}
\usepackage{multicol}
\usepackage{url}

\title{Homework, Week 2, part B}
\date{October 24, 2013}

\begin{document}
\maketitle{}

\paragraph{}The software development process has four steps. This week's homework will walk you through all four steps. The steps are: (1) analysis, (2) design, (3) implementation, and (4) testing. The process repeats all four steps, one after another, many times before the software is complete.

\paragraph{Implementation}The purpose of implementation is to write the software using the design. Barski's game is at \url{http://landoflisp.com/guess.lisp}. I have copied the code below. Copy this into a file named ``guess.lisp'', load it into your clisp environment, and run it.

\lstset{language=Lisp,numbers=left,keepspaces=true,basicstyle=\small,numberstyle=\tiny}
\begin{lstlisting}
;;;guess.lisp
(defparameter *small* 1)
(defparameter *big* 100)

(defun guess-my-number ()
     (ash (+ *small* *big*) -1))

(defun smaller ()
     (setf *big* (1- (guess-my-number)))
     (guess-my-number))

(defun bigger ()
     (setf *small* (1+ (guess-my-number)))
     (guess-my-number))

(defun start-over ()
   (defparameter *small* 1)
   (defparameter *big* 100)
   (guess-my-number))
\end{lstlisting}

\paragraph{Testing}Load this file into your lisp with \texttt{(load "guess.lisp")} and evaluate \texttt{(guess-my-number)} or \texttt{(start-over)}. The computer will guess 50, and you will enter \texttt{(bigger)} or \texttt{(bigger)} to have the computer make another guess. Does it do what it is supposed to do? Does it meet all the requirements you identified? Does it do anything that you did not identify as a requirement?

\end{document}
