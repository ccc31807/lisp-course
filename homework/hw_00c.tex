\documentclass{article}

\usepackage[top=1in]{geometry}
\usepackage{listings}
\usepackage{multicol}

\title{Homework, Week 0, part C}
\date{September 12, 2014}

\begin{document}
\maketitle{}
\lstset{language=Lisp}

Invoke clisp and type the following commands. Remember, a `:q' (COLON QUE), `:a' (COLON AIGH), CONTROL-BREAK, or CONTROL-C will restore the top level prompt if you make a mistake. This explores the Lisp functions, \texttt{(minusp), (zerop), (plusp), (evenp), (oddp), (rem), (mod), (random), and (ash)}. Write down each result and see if you can figure out what each one does.

Functions that end in the letter ``p'' are called \textit{predicates}. Predicates return either TRUE or FALSE. This homework also includes the remainder (or modulus) operations. \texttt{(random)} is a random number generator, useful for games of chance. \texttt{(ash)} is the arithmetic shift function. It takes two arguments, the number to be shifted, and a number indicating the amount to shift -- to the left for a positive amount, and to the right for a negative number -- effectively doubling or halving the number given as the first argument.

\begin{multicols}{2}
\begin{lstlisting}
(minusp -1)
(minusp 0)
(minusp 1)
(zerop -1)
(zerop 0)
(zerop 1)
(plusp -1)
(plusp 0)
(plusp 1)

(evenp 4)
(evenp 5)
(oddp 6)
(oddp 7)

;; run with (rem) and (mod)
(rem 4 9)
(rem 5 9)
(rem 6 9)
(rem 7 9)
(rem 8 9)
(rem 0 9)
(rem 9 4)
(rem 9 5)
(rem 9 6)
(rem 9 7)
(rem 9 8)
(rem 9 9)
(rem 9 4)

;; run these several times
;What is the difference?
(random 1.0)
(random 1)
(random 11)
(random 11.0)
(random 101)
(random 101.0)

(ash 4 0)
(ash 4 1)
(ash 4 2)
(ash 4 3)
(ash 4 4)
(ash 28 0)
(ash 28 -1)
(ash 28 -2)
(ash 28 -3)
(ash 28 -4)
\end{lstlisting}
\end{multicols}

\end{document}
