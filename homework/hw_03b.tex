\documentclass{article}

\usepackage{listings}
\usepackage{multicol}
\usepackage{url}

\title{Homework, Week 3, part B}
\date{October 31, 2013}

\begin{document}
\maketitle{}

\paragraph{}This assignment uses the list you wrote yesterday. Write a function. Name it \texttt{(list-answers)}. It will take as an argument a list of answers, like this: \texttt{("George Washington" "John Adams" "Thomas Jefferson" "James Madison")}. It will list the answers randomly, that is, it will not list them in order. If the first answer is always the correct answer, it would not do much good on a test to always list the right answer first.

\paragraph{}There are many ways to do this. My answer is below. I do not expect you to write this by yourself, because you have not had all you need. I do expect you to think about it, and write what you can in plain English as if a computer could understand it. Write down all the steps. This is very important. If you can't think through all the steps and write them in English, you cannot think through all the steps and write them in Lisp, or in any other language.

\lstset{language=Lisp,numbers=left,keepspaces=false,basicstyle=\small,numberstyle=\tiny,breaklines=true,showstringspaces=false}
\begin{lstlisting}
;;my output
[174]> (defparameter answers (cadr (nth 4 items)))
ANSWERS
[175]> answers
("Ron Reagan" "Richard Nixon" "John Kennedy" "Harry Truman")
[176]> (list-answers answers)
    Harry Truman
    Richard Nixon
    Ronald Reagan
    John Kennedy
NIL
\end{lstlisting}

\lstset{language=Lisp,numbers=left,keepspaces=false,basicstyle=\small,numberstyle=\tiny,breaklines=true,showstringspaces=false}
\begin{lstlisting}
;;my function
(defun list-answers (answers)
  (cond
    ((null answers) nil)
    (t (let* ((len (length answers))
             (num (random len)))
         (format t "    ~a~%" (nth num answers))
         (list-answers (remove (nth num answers) answers))))))
\end{lstlisting}

\end{document}
