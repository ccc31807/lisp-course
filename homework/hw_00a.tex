\documentclass{article}

\usepackage[top=1in]{geometry}
\usepackage{listings}
\usepackage{multicol}

\title{Homework, Week 0, part A}
\author{lightprogramming.org}
\date{\today}

\begin{document}
\maketitle{}
\lstset{language=Lisp}

Invoke clisp and type the following commands. Remember, a `:q' (COLON QUE), `:a' (COLON AIGH), CONTROL-BREAK, or CONTROL-C will restore the top level prompt if you make a mistake. This explores five Lisp functions, + (addition), - (subtraction), * (multiplication), / (division), and (expt) which is short for `exponent'. Write down each result and see if you can figure out what each one does. Some will create errors, such as \texttt{(-)}.

The purpose of this lab is to explore arithmetic operators: addition, subtraction, multiplication, and division. Division can produce a ratio (fraction), such as one-third ($\frac{1}{3}$), or a decimal number, such as 3.33333. You should discover how to produce both ratios and decimal numbers. This lab also introduces exponentiation: $2^2 = 4$, $2^3 = 8$, and so on. Optional: what happens if the exponent is a ratio (fraction)? a decimal fraction? a negative integer? or a negative decimal?
\begin{multicols}{2}
\begin{lstlisting}
(+)
(+ 1)
(+ 1 2)
(+ 1 2 3)
(+ 1 2 3 4)

(-) ;this is an error
(- 4)
(- 4 3)
(- 4 3 2)
(- 4 3 2 1)

(*)
(* 1)
(* 1 2)
(* 1 2 3)
(* 1 2 3 4)

(/)
(/ 4)
(/ 4 3)
(/ 4 3 2)
(/ 4 3 2 1)

;;do these give the same answer?
(/ 7 21)
(/ 7.0 21)
(/ 7 21.0)

(expt 2 0)
(expt 2 1)
(expt 2 2)
(expt 2 3)
(expt 2 4)
(expt 2 5)
(expt 100 2)
(expt 2 100)
;;this will impress you!
(expt 99 99)
\end{lstlisting}
\end{multicols}

\paragraph{}Exit from clisp with \texttt{(QUIT)}.
\end{document}
