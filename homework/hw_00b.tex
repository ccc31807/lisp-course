\documentclass{article}

\usepackage[top=1in]{geometry}
\usepackage{listings}
\usepackage{multicol}

\title{Homework, Week 0, part B}

\date{September 12, 2014}
\begin{document}
\maketitle{}
\lstset{language=Lisp}

Invoke clisp and type the following commands. Remember, a `:q' (COLON QUE), `:a' (COLON AIGH), CONTROL-BREAK, or CONTROL-C will restore the top level prompt if you make a mistake. This explores six Lisp functions, =, /=, \textgreater, \textgreater=, \textless, and \textless=. Write down each result and see if you can figure out what each one does. Some will create errors.

These are all functions, and they all return either TRUE or FALSE. In Common Lisp, TRUE is represented by ``\texttt{T}'', and FALSE is represented either by ``\texttt{NIL}'' or by () -- an empty set of parentheses. \textit{Important}: in logic, the truth or falsity of a statement is many times of critical importance. Make sure you understand \textit{why} these commands evaluate to TRUE or FALSE.


\begin{multicols}{2}
\begin{lstlisting}
(= 9)
(= 9 9)
(= 9 9 9)
(= 9 9 9 8)

(/= 9)
(/= 9 8)
(/= 9 8 7 6)
(/= 9 9 9 6)

(< 9 8 7 6)
(< 6 7 8 9)
(< 6 6 7 8)

(<= 9 8 7 6)
(<= 6 7 8 9)
(<= 6 6 7 8)

(> 9 8 7 6)
(> 6 7 8 9)
(> 9 9 8 7)

(>= 9 8 7 6)
(>= 6 7 8 9)
(>= 9 9 8 7)
\end{lstlisting}
\end{multicols}

\paragraph{} Exit from clisp by entering \texttt{(QUIT)}.

\end{document}
