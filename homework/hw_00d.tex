\documentclass{article}

\usepackage{amsmath}
\usepackage{listings}
\usepackage{multicol}

\title{Homework, Week 0, part D}
\date{September 12, 2014}

\begin{document}
\maketitle{}
\lstset{language=Lisp}

\paragraph{}Invoke clisp and type the following commands. Remember, a `:q' (COLON QUE), `:a' (COLON AIGH), CONTROL-BREAK, or CONTROL-C will restore the top level prompt if you make a mistake. This explores five Lisp functions, + (addition), - (subtraction), * (multiplication), and / (division), and (expt) which is short for `exponent'. Write down each result and see if you can figure out what each one does.\\

\begin{lstlisting}
;;type these in clisp
;;then, do them by hand
(+ (* 3 5) (* 4 2))
(/ (+ 4 3) (* 6 2))
(/ (* (- 7 3) (+ 2 2)) 3)
(* (+ 5 4) (* 2 4) (- 13 7) (/ 36 12))
(/ (+ (* 6 4) (* 3 4)) (- (* 9 11) (* 8 7)))

;;do these by hand
;;then, type them in clisp
\end{lstlisting}
$(5 + 8) \times (18 - 10)$ \\
\\
$(5 + 4) \times (3 - 2) \div (99 + 99)$\\
\\
$\frac{(4 \times 3) + (10 + 2)}{(20 - 13) + (2 + 2)}$ \\
\\
$\frac{(4 \times 3) + (10 + 2) - (+ 5 5)}{(20 - 13) + (2 + 2) \times -1}$ \\
\\
$\frac{1}{( 4 \:+ \: 3) \times (10 \:- \:    3)}$ \\
\\
HINT: \texttt{(expt 4 -1)} is the same thing as dividing 1 by 4. In other words, \texttt{(expt (- 7 3) -1)} is the same thing as $\frac{1}{7 - 3}$ or $1 \div (7 - 3)$. \\


\paragraph{}Exit from clisp with \texttt{(QUIT)}.
\end{document}
