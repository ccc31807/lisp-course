\documentclass{article}

\usepackage{listings}
\usepackage{multicol}
\usepackage{url}
\usepackage{textcomp}

\title{Homework, Week 8, part D}
\date{December 20, 2013}

\begin{document}
\maketitle{}

\paragraph{}Implement Barski's code on pages 116 -- 125 of the book. The purpose of this is to build a utility, which Barski calls `graph-utils.lisp' for use in the next chapter of the book. I do not agree with the way that Barski presents this material, but since he has not covered some useful functions, he is doing the best that he can. I rewrote this code, but after looking at it, I think that Barski's code ls less confusing than mine would be to you.

\paragraph{}Barski talks about something that he calls a `thunk.' I had never heard of thunks before I read his book, and I have never seen a thunk outside of his book. However, I use the technique a lot, you you will, too. The concept is very simple: send the output of a program to the screen until you get it right, then send the output to a document. In my opinion, Barski's approach is needlessly complex, and I do not think that you need to understand it. However, it will certainly do no harm to understand it, and you at least need to think about what he is doing in his code.

\end{document}
