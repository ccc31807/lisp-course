\documentclass{article}

\usepackage[top=1in]{geometry}
\usepackage{listings}
\usepackage{multicol}

\title{Homework, Week 2, part D}
\date{October 24, 2013}

\begin{document}
\maketitle{}

A computer's memory consists of numbered cells, billions of them. The numbers are hexadecimal numbers, and raw memory addresses look like this: 35a6bf95468a20c2. Because the memory addresses are so hard to remember, we can give individual memory cells names, like \textit{sum} or \textit{count}. We can then place \textit{values} in memory cells, and access the values by calling the name of the memory cell. We call this name-value combination a \textit{variable}.

More formally, we speak of \textit{binding} a value to a variable. When the variable has a value, we say that the variable is \textit{bound}. When the variable does not have a value, we say that the variable is \textit{unbound}.

Variables that can be seen and accessed throughout the entire program are called \textit{global} variables. Variables that can be seen only in a part of the program are called \textit{lexical} variables, or sometimes \textit{local} variables. Lisp has a number of ways of binding variables to values. Evaluate the following statements. Then, look up the functions and read the function definitions.
\lstset{language=Lisp,numbers=left,keepspaces=true,basicstyle=\small,numberstyle=\tiny,showstringspaces=false,breaklines=true}
\begin{lstlisting}
;; global variable bindings
;; we will mostly use defparameter
(defparameter a 2)
(defvar b 4)
(defconstant c 6)
(+ a b c)

;; lexical variable bindings
;; we will mostly use setf
(setf d 3)
(setq e 5)
(set 'f 7)
(+ d e f)

;; lexical variable bindings
;;   used inside functions for
;;   temporary variables
;; also see another version, let*
(let 
  ((g 10)
   (h 20)
   (i 30))
  (+ g h i)) 
\end{lstlisting}

\end{document}
